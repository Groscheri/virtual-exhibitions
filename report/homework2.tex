\documentclass[a4paper,11pt]{article}
\usepackage[utf8]{inputenc}
\usepackage[english]{babel}
\usepackage[font=small,labelfont=bf, textfont=it]{caption}
\usepackage[bottom=2cm, left=3.7cm, right=3.7cm]{geometry} % to change padding
\usepackage{verbatim}
\usepackage{subcaption} % for multi figure
\usepackage{enumitem} % -- label item
\usepackage{tabularx}
\usepackage{color}
\usepackage[usenames, dvipsnames]{xcolor} % color
\usepackage[framemethod=TikZ]{mdframed} % box
\usepackage{listings} % code
\usepackage{amsmath} % align* maths
\usepackage{wrapfig}
\usepackage[bookmarks, hidelinks]{hyperref} % clickable links
\usepackage{graphicx} % includegraphics
\usepackage[section]{placeins} % float inside section
\usepackage{indentfirst}

\setlength\parindent{0pt}
\setlength{\parskip}{1em}



% Titre
\title{Distributed Artificial Intelligence \& Intelligent Agents \\ DAIIA - Homework 2}
\author{Laurentiu CAPATINA \& Quentin LEMAIRE}

\begin{document}

  \maketitle % build title

  \section{Introduction}
  
  The aim of this homework is to go deeper in the mechanisms of the JADE\footnote{Java Agent DEvelopment: \href{http://jade.tilab.com/}{http://jade.tilab.com/}} 
  platform. 
  Based on a virtual exhibition scenario, we will implement agents and behaviors which will communicate with each other. We will also implement Dutch auction 
  between 4 different agents (1 auctioneer \& 3 bidders). Finally, a theoretical analysis will be done in order to find Nash Equilibrium within a given problem.
  
  % TODO remove these 2 subsections ??
  \subsection{Agents}

  % TODO agents in this homework

  \subsection{Behaviors}
 
  % TODO behaviors in this homework

  % aliases
  \newcommand{\pa}{\textit{Profiler Agent}}
  \newcommand{\cu}{\textit{Curator Agent}}
  \renewcommand{\to}{\textit{Tour Guide Agent}}
  
  \section{Implementation}
  
  \subsection{Communication}

  A scenario has been implemented using different \textit{ACLMessage}s between each agent. The aim of this scenario is to provide a virtual tour to the \pa{}. This 
  virtual tour is generated by the \to{} which requests the \cu{} in order to get information about artifacts in the museum. We have the following communication:
  \begin{enumerate}
   \item The \pa{} asks the \to{} to generate a virtual tour given the interests of the user the agent is representing.
   \item The \to{} receives the request and asks the \cu{} information about items inside the museum.
   \item The \cu{} receives the request and responds with the requested information to the \to{}.
   \item The \to{} gets information and builds a personalized virtual tour for the \cu{}. The agent then answers its request (1.).
   \item The \pa{} receives the virtual tour it asked for but it doesn't contain any information about artifacts, this is only technical IDs. The agent has to ask 
   the \cu{} information about these items.
   \item The \cu{} answers the \pa{} with the information of each artifacts contained in the virtual tour.
  \end{enumerate}
  
  This scenario has been implemented with the use of different classes and behaviors. First, a \textit{SimpleREAchieveInitiator} behavior is used to create the 
  first request of the \pa{}. A \textit{SimpleREAchieveResponder} behavior is done by the \to{} to handle the first request. When this agent receives the request, 
  it creates an ACLMessage to request the \cu{}. It then blocks its execution to wait for the answer (using \textit{blockingReceive} method). The answer received, 
  the \textit{prepareResultNotification} method of the \textit{SimpleREAchieveResponder} terminates responding to the \pa{}. The \pa{} uses a \textit{SimpleREAchieveInitiator} 
  behavior in order to get information about the artifacts inside the virtual tour.

  
%   1. Create a CuratorAgent called "cucu" without any argument
% CuratorAgent will start listening to TourGuide & Profiler agents (info requests)
% 
% 2. Create a TourGuideAgent called "toto" without any argument
% TourGuideAgent will start listening to Profiler agent (build tour requests)
% 
% 3. Create a ProfilerAgent called "pro" without any argument
% Wait at least 2 seconds, the ProfilerAgent will wake up and send a "build tour" 
% request to TourGuideAgent. Then, the TourGuide agent will make an info request 
% to CuratorAgent in order to get information about artifacts in the virtual 
% museum. When it receives information about artifacts, it builds the virtual tour 
% and answers the ProfilerAgent.
% Finally, the ProfilerAgent send an info request to the Curator agent to get 
% information about items it received (from TourGuide agent) in the virtual tour.

  \subsection{Dutch auction}

  % TODO

  \section{Nash Equilibrium}
  
  \subsection{Problem}

  In this scenario, we have different actors:
  \begin{itemize}
   \item An Artist Management agent
   \item Several Curator agents
   \item A Profiler agent
  \end{itemize}
  
  The Artist Management agent is trying to sell some paintings. Those paintings have 
  different quality (high or low) and are sold at the same price! That's why it is 
  ``mandatory'' for this agent to sell low quality paintings in order to increase its 
  payoff.
  
  The Curator agents act as retailers. They don't know about the quality of the paintings 
  they are trying to ``sell'' on the Artist Management agent's behalf. These agents are 
  trying to modify the price of the paintings in order to earn money and they are trying to 
  correctly focus the right profiler agent according to its interests. An other strategy could 
  be to adapt the price according to the demand of the market. The more valuable a painting is, 
  the more it can be expensive.
  
  Finally, the profiler agent is trying to buy paintings. But it doesn't known neither the 
  quality of the paintings until it buys them. The aim of this agent is to buy high quality 
  paintings at a low price (best payoff).

  \subsection{Utilities}

  \subsubsection{Artist Management agent}
  
  The seller (Artist Management agent) has 4 possible issues:
  \begin{itemize}
   \item sell a low quality product (payoff = 3)
   \item sell a high quality product (payoff = 1)
   \item don't sell a low quality product (payoff = -1)
   \item don't sell a high quality product (payoff = -3)
  \end{itemize}
  
  2 actions are possible for this agent:
  \begin{itemize}
   \item Produce high quality product
   \item Produce low quality product
  \end{itemize}
  
  \subsubsection{Curator agents}

  These agent won't lower the price of the painting. At most, they are going 
  to increase the price but this won't have any impact on the issue of the problem.
  Curator agents will earn more or less money according to the price they put and if 
  they increase too much the price, they won't be able to sell anything. This behavior 
  impacts the probability for the Artist Management agent to sell its painting.
  
  That's why the decisions taken by the curator agents don't have any impact in order to 
  find Nash Equilibrium. But of course, in real life, this has a huge impact because it 
  influences the possible result of the problem.
  
  \subsubsection{Profiler agent}
  
  The buyer (Profiler agent) has 4 possibles issues:
  \begin{itemize}
   \item buy a high quality product (payoff = 3)
   \item buy a low quality product (payoff = 0)
   \item don't see (i.e. don't buy) a low quality product (payoff = 0)
   \item don't see (i.e. don't buy) a high quality product (payoff = -3)
  \end{itemize}
  
  2 actions are possible for this agent:
  \begin{itemize}
   \item Buy \& see the quality of the product
   \item Don't buy \& don't see the quality of the product
  \end{itemize}
  
  \subsection{Payoff matrix}
  
  Based on the result above, we can display the payoff matrix.
  
  \begin{tabular}{|c|c|c|}\hline
   (Seller, Buyer) & buy \& see & don't buy \& don't see \\\hline
   Produce a HIGH quality product & (1,3) & (-3,-3) \\\hline
   Produce a LOW quality product & \color{red}{(3,0)} & \color{red}{(-1,0)} \\\hline
  \end{tabular}

  \color{red}{Red values}\color{black}{} corresponds to Nash Equilibrium.

  % TODO to check !!! THIS IS PROBABLY TOTALLY WRONG

  \section{Conclusion}
  
  % TODO

\end{document}
