\documentclass[a4paper,11pt]{article}
\usepackage[utf8]{inputenc}
\usepackage[english]{babel}
\usepackage[font=small,labelfont=bf, textfont=it]{caption}
\usepackage[bottom=2cm, left=3.7cm, right=3.7cm]{geometry} % to change padding
\usepackage{verbatim}
\usepackage{subcaption} % for multi figure
\usepackage{enumitem} % -- label item
\usepackage{tabularx}
\usepackage{color}
\usepackage[usenames, dvipsnames]{xcolor} % color
\usepackage[framemethod=TikZ]{mdframed} % box
\usepackage{listings} % code
\usepackage{amsmath} % align* maths
\usepackage{wrapfig}
\usepackage[bookmarks, hidelinks]{hyperref} % clickable links
\usepackage{graphicx} % includegraphics
\usepackage[section]{placeins} % float inside section
\usepackage{indentfirst}

\setlength\parindent{0pt}
\setlength{\parskip}{1em}



% Titre
\title{Distributed Artificial Intelligence \& Intelligent Agents \\ DAIIA - Homework 2}
\author{Laurentiu CAPATINA \& Quentin LEMAIRE}

\begin{document}

  \maketitle % build title

  \section{Introduction}
  
  The aim of this homework is to go deeper in the mechanisms of the JADE\footnote{Java Agent DEvelopment: \href{http://jade.tilab.com/}{http://jade.tilab.com/}} 
  platform. 
  Based on a virtual exhibition scenario, we will implement agents and behaviors which will communicate with each other. We will also implement Dutch auction 
  between 4 different agents (1 auctioneer \& 3 bidders). Finally, a theoretical analysis will be done in order to find Nash Equilibrium within a given problem.
  
  % TODO remove these 2 subsections ??
  \subsection{Agents}

  % TODO agents in this homework

  \subsection{Behaviors}
 
  % TODO behaviors in this homework

  % aliases
  \newcommand{\pa}{\textit{Profiler Agent}}
  \newcommand{\cu}{\textit{Curator Agent}}
  \renewcommand{\to}{\textit{Tour Guide Agent}}
  \newcommand{\am}{\textit{Artist Manager Agent}}
  
  \section{Implementation}
  
  \subsection{Communication}

  A scenario has been implemented using different \textit{ACLMessage}s between each agent. The aim of this scenario is to provide a virtual tour to the \pa{}. This 
  virtual tour is generated by the \to{} which requests the \cu{} in order to get information about artifacts in the museum. We have the following communication:
  \begin{enumerate}
   \item The \pa{} asks the \to{} to generate a virtual tour given the interests of the user.
   \item The \to{} receives the request and asks the \cu{} information about items inside the museum.
   \item The \cu{} receives the request and responds with the requested information to the \to{}.
   \item The \to{} gets information and builds a personalized virtual tour for the \cu{}. The agent then answers its request (1.).
   \item The \pa{} receives the requested virtual tour but it doesn't contain any information about artifacts. They are represented only by their IDs. The agent has to ask 
   the \cu{} for additional information about these items.
   \item The \cu{} eventually provides the \pa{} the requested information for each artifact of the virtual tour.
  \end{enumerate}
  
  This scenario has been implemented with the use of different classes and behaviors. First, a \textit{SimpleREAchieveInitiator} behavior is used to create the 
  first request of the \pa{}. A \textit{SimpleREAchieveResponder} behavior is done by the \to{} to handle the first request. When this agent receives the request, 
  it creates an ACLMessage to request the \cu{}. It then blocks its execution while waiting for the answer (using \textit{blockingReceive} method). The answer received, 
  the \textit{prepareResultNotification} method of the \textit{SimpleREAchieveResponder} terminates with the response to the \pa{}. The \pa{} uses a \textit{SimpleREAchieveInitiator} 
  behavior in order to get information about the artifacts inside the virtual tour.

  
%   1. Create a CuratorAgent called "cucu" without any argument
% CuratorAgent will start listening to TourGuide & Profiler agents (info requests)
% 
% 2. Create a TourGuideAgent called "toto" without any argument
% TourGuideAgent will start listening to Profiler agent (build tour requests)
% 
% 3. Create a ProfilerAgent called "pro" without any argument
% Wait at least 2 seconds, the ProfilerAgent will wake up and send a "build tour" 
% request to TourGuideAgent. Then, the TourGuide agent will make an info request 
% to CuratorAgent in order to get information about artifacts in the virtual 
% museum. When it receives information about artifacts, it builds the virtual tour 
% and answers the ProfilerAgent.
% Finally, the ProfilerAgent send an info request to the Curator agent to get 
% information about items it received (from TourGuide agent) in the virtual tour.

  \subsection{Dutch auction}

  \subsubsection{Description of the Dutch Auction}
  
  This auction has one auctioneer and several bidders. In contrast to the well-known English 
  auction, it starts with a very high price and it will slowly decrease until one of the bidders 
  claims to buy the object. Once an offer was accepted by a bidder, the auction ends. However, 
  it also has a minimum price: if the price of the auction goes below this value, it will end without being 
  sold. The Dutch auction is susceptible at the winner's curse. In other words, the winner has high 
  chances to pay more than the the esitmated price in order to be assured to get in its possession.
  
  \subsubsection{Actors involved in the auction}
  
  The Dutch auction was implemented for the selling of the paintings. The seller in our chosen case 
  is the \am{} and the buyers are one or several Curators. By sending arguments at creation of an 
  \am{}, we can define the first price of the auction, the amount by which the price is decreased 
  at each step and the minimal price acceptable for the auction. These characteristics of the Artist 
  Manager ar not known by the Curators and they discover them during the auction. On the other hand,
  the \cu{} can be parameterzied by choosing one of the following strategies:
  \begin{enumerate}
   \item Buy the auctioned item after \textit{x} numbers of offers
   \item Clain the object after the value reached \textit{x}\% of the first proposed price
   \item A certain amount is fixed which can pe spent in case the object reaches that particular value
   \item Buy the object if the value reached \textit{x}\% of the price privously estimated by Curator
  \end{enumerate}

  
  
  \subsubsection{Implementation of auction using JADE}
  
  The \am{} initializes an auction by sending a first message to all Curators previously identified 
  on the platform. The offer uses the performative \textit{ACLMessage.CFP} in order to outline that 
  the message contains an offer. Using the \textit{blockingReceive} method, the \am{} waits for a response. 
  After the expiration of the 3s timeout, a new offer is calculated and sent using the same performative. 
  Meanwhile, the \cu{} listens for messages which come from an \am{}. In case the performative is 
  \textit{ACLMessage.CFP}, he will evaulate the price according to his strategy. If he can afford to 
  pay, an ACLMessage is sent to the \am{}. When the object is sold the one \cu{}, a message with the 
  performative \textit{ACLMessage.CONFIRMATION} is sent to him and a message with the performative 
  \textit{ACLMessage.CANCEL} to the others. As privously stated, the auction can end as
  well without selling the object if the price goes under the minimal value specified by the \am{}.

  \section{Nash Equilibrium}
  
  \subsection{Problem}

  In this scenario, we have 3 different actors:
  \begin{itemize}
   \item One Artist Manager agent
   \item Several Curator agents
   \item A Profiler agent
  \end{itemize}
  
  The \am{} is trying to sell the paintings produced by his artist. Those paintings have 
  different types of quality (high or low) but they are sold at the same price! Consequently, it is desireable 
  for this agent to sell low quality paintings due to the increase in payoff.
  
  The \cu{}s act as retailers. They don't know about the quality of the paintings 
  they are trying to ``sell'' on behalf of the Artist Manager agent. These agents are 
  trying to increase the price of the paintings in order to make profit. They are also trying to 
  correctly focus on the right profiler agent according to his interests. Another strategy could 
  be to adapt the price according to the demand of the market. The more valuable a painting is, 
  the more it can be expensive.
  
  Finally, the profiler agent is trying to buy paintings. However he is not aware of the 
  quality of the paintings until he buys them. The aim of this agent is to purchase high quality 
  paintings at the lowest possible price (best payoff).

  \subsection{Utilities}

  \subsubsection{Artist Manager agent}
  
  The seller (\am{}) has 4 possible issues:
  \begin{itemize}
   \item sell a low quality product (payoff = 3)
   \item sell a high quality product (payoff = 1)
   \item don't sell a low quality product (payoff = -1)
   \item don't sell a high quality product (payoff = -3)
  \end{itemize}
  
  2 actions are possible for this agent:
  \begin{itemize}
   \item Produce high quality product
   \item Produce low quality product
  \end{itemize}
  
  \subsubsection{Curator agents}

  These agents won't lower the price of the painting.
  Curator agents will earn more or less money according to the price they set and if 
  they increase it too much, they won't be able to sell anything. This behavior 
  impacts the probability for the \am{} to sell its painting.
  
  That's why the decisions taken by the curator agents don't have any impact in order to 
  find Nash Equilibrium. But of course, in real life, this has a huge impact because it 
  influences the possible result of the problem.
  
  \subsubsection{Profiler agent}
  
  The buyer (Profiler agent) has 4 possibles issues:
  \begin{itemize}
   \item buy a high quality product (payoff = 3)
   \item buy a low quality product (payoff = 0)
   \item don't see (i.e. don't buy) a low quality product (payoff = 0)
   \item don't see (i.e. don't buy) a high quality product (payoff = -3)
  \end{itemize}
  
  2 actions are possible for this agent:
  \begin{itemize}
   \item Buy \& see the quality of the product
   \item Don't buy \& don't see the quality of the product
  \end{itemize}
  
  \subsection{Payoff matrix}
  
  Based on the result above, we can display the payoff matrix. This specific matrix corresponds to the case when the \cu{} doesn't gain
  any profit (i.e. sells the object at the same price as bought).
  
  \begin{tabular}{|c|c|c|}\hline
   (Seller, Buyer) & buy \& see & don't buy \& don't see \\\hline
   Produce a HIGH quality product & (1,3) & (-3,-3) \\\hline
   Produce a LOW quality product & \color{red}{(3,0)} & \color{red}{(-1,0)} \\\hline
  \end{tabular}

  \color{red}{Red values}\color{black}{} corresponds to Nash Equilibrium.

  As previously stated, the curator can increase the price which will impact on the gain of the buyer (i.e. \pa{}) and on his
  probability to buy or not the product. Consequently, we suppose for example a \textit{10\%} of the price. 
  This case gives us the following matrix:
  
   \begin{tabular}{|c|c|c|}\hline
   (Seller, Buyer) & buy \& see & don't buy \& don't see \\\hline
   Produce a HIGH quality product & (1,2) & (-3,-2) \\\hline
   Produce a LOW quality product & (3,-1) & \color{red}{(-1,1)} \\\hline
  \end{tabular}
  
  \color{red}{Red values}\color{black}{} corresponds to Nash Equilibrium.
  
  We can observe that the \cu{} has an impact on the Nash Equilibrium. If the price are increased more, the gain for the buyer
  will evolve in the same direction as for this case.

  \section{Conclusion}
  
  First of all, this homework gave us the opportunity to explore both the implementation of JADE for the transmission of messages.
  We applied this framework in a practical case, a Dutch auction with several participating bidders who apply different
  strategies. The last part, gave us a better understanding of the concept of Nash Equilibrium and we applied it in the case
  of a buying/selling situation.

\end{document}
