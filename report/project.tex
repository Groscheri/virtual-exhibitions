\documentclass[a4paper,11pt]{report}
\usepackage[utf8]{inputenc}
\usepackage[english]{babel}
\usepackage[font=small,labelfont=bf, textfont=it]{caption}
%\usepackage[top=1.5cm, bottom=1.5cm, left=2cm, right=2cm]{geometry} % to change padding
\usepackage{verbatim}
\usepackage{subcaption} % for multi figure
\usepackage{enumitem} % -- label item
\usepackage{tabularx} % table
\usepackage{color}
\usepackage[usenames, dvipsnames]{xcolor} % color
\usepackage[framemethod=TikZ]{mdframed} % box
\usepackage{listings} % code
%\usepackage{amsmath} % align* maths
\usepackage[bookmarks, hidelinks, linktoc=all]{hyperref} % clickable links
\usepackage{graphicx} % includegraphics
\usepackage{indentfirst} % indentation
\usepackage{titlesec} % spacing between sections
\usepackage{multicol} % multi columns
\usepackage{tikz}
\usetikzlibrary{shapes,arrows}
% \usepackage{tocloft}
%\setlength\cftaftertoctitleskip{50pt} % toc space after title
%\setlength{\cftbeforetoctitleskip}{-3em} % toc space before title

\setlength\parindent{0pt}
\setlength{\parskip}{0.9em} % paragraph vertical space

\renewcommand\thesection{\arabic{section}} % start section from 1
\setcounter{tocdepth}{1} % display subsection in toc
\setcounter{secnumdepth}{3} % number subsubsection

% space before lists
%\setlist[itemize]{topsep=0pt}
%\setlist[enumerate]{topsep=0pt}

% spacing between figure and caption
% \setlength{\abovecaptionskip}{2pt plus 2pt minus 4pt} % default: 10pt

% Titre
% \title{Distributed Artificial Intelligence \& Intelligent Agents \\ DAIIA - Project}
% \author{Laurentiu CAPATINA \& Quentin LEMAIRE}

\begin{document}

  % first page
  \pagenumbering{Alph}
\begin{titlepage}
  \centering
  \vspace*{\stretch{1}}
  \vfill
    {\bfseries\Large{
	Distributed Artificial Intelligence\\\&\\Intelligent Agents}
    }
  \vfill
  \vfill
    \Huge{\textsc{Smart Museum}}
    \\\vspace{10pt}
    \Large{Project report}
  \vfill
      \Large{\textsc{Laurentiu Capatina} \& \textsc{Quentin Lemaire}}
    \\
  \vspace{0.4cm}
    Group 5
  \vfill
  \vfill
    % \includegraphics[width=0.22\textwidth]{media/kth.jpg}
    KTH Royal Institute of Technology
  \vfill
    \today
  \vspace*{\stretch{1}}
  \thispagestyle{empty}
\end{titlepage}
\pagenumbering{arabic}

\addtocontents{toc}{\protect\thispagestyle{empty}} % remove toc pagination
\tableofcontents{} % toc
\clearpage % leave a page
\setcounter{page}{1} % init counter page

  \section{Introduction}
  % TODO describe aim of the project
  % TODO describe briefly different tasks
  
  This project deals with \textbf{Agent Oriented Software Engineering} (AOSE) in the 
  context of a virtual exhibitions (smart museum). The aim is to model a multi-agent 
  system with specific methods of modelisation: GAIA and Role-base modeling.
  As a consequence of the previous works, this project reuses concepts we learnt 
  during lectures and homeworks: behaviors, messaging, negotiation, mobility, ...
  
  
  %%%%%%
  
  
  \section{GAIA Methodology} % task 1
  % Model your system via GAIA AOSE Methodology
  % Agent Oriented Sofware Engineering (AOSE)
  
  \subsection{Requirements statement}
  % TODO text describing the scenario
  
  In the smart museum scenario, we have different actors: artist managers, curators 
  and profilers. Each actor interacts with each other and has different tasks to 
  perform.
  
  Firstly, artist manager is responsible to gather paintings or art creations 
  from a set of different artists he or she is in contact with. The goal of the manager 
  is to sell products with different quality level to profilers.
  
  In order to sell paintings to profilers, artist manager has to use curators as retailers 
  because they don't know profilers directly. Thus, they have to provide products with an 
  announced price to curators which will be responsible to sell it to profilers. Because there 
  are several curators and only one curator can retail one product, curators have to come and 
  visit artist manager in order to perform a Dutch auction. This auction will determine which 
  curator will be asked to retail the product.
  
  Finally, Profiler can explore different products offered by curators and tell whether he or 
  she wants to buy the product at the displayed price.
  
  % TODO put schemes (the one in instructions ?)
  
  \subsection{Roles model}
  % TODO tables
  
  Different roles can be modeled in our scenario:
  \begin{itemize}
   \item Artist Manager role (described in Figure~\ref{figure:role_artist_manager})
   \item Curator role (described in Figure~) % TODO
   \item Profiler role (described in Figure~) % TODO
  \end{itemize}

  % TODO
  \begin{figure}[ht!]
    \begin{mdframed}
      Role Schema: \textsc{Artist Manager} \\ \hrule \vspace{2pt} \hrule \vspace{10pt}
      Description:\\
      Gathers high or low quality products and acts as auctioneer with Curators.
      \\ \hrule \vspace{10pt}
      Protocols and Activities:
      \vspace{-10pt}
      \begin{flushleft}
       \underline{ChooseHighOrLowQualityProduct}, \underline{WaitTwoOrMoreCurators}, DutchAuction, \underline{WaitMoneyOrProductBack}
      \end{flushleft}
      \hrule \vspace{10pt}
      Permissions:\\
      \vspace{-10pt}
      \begin{center}
       reads \textit{paintingName //name of the painting}\\
	    \textit{paymentReceived //payment received for a painting}\\
       changes \textit{paintingPrice //price of painting}
      \end{center}
      % ask "artist" to product high or low quality product
      % auction product
      \hrule \vspace{10pt}
      Responsabilities:\\
      Liveness:
      \vspace{-10pt}
      \begin{flushleft}
      \small\textsc{Artist Manager} = (\underline{ChooseHighOrLowQualityProduct}, \underline{WaitTwoOrMoreCurators}, DutchAuction, \underline{WaitMoneyOrProductBack})$^\omega$
      \end{flushleft}
      Safety:
      \vspace{-10pt}
      \begin{itemize}
       \item $Money > 0$
       \item $Paintings >= 0$
      \end{itemize}
 % TODO
    \end{mdframed}
  \caption{Artist manager role}
  \label{figure:role_artist_manager}
  \end{figure}
  
  \subsection{Interaction Model}
  % TODO tables
  
  \subsection{Agent Model}
  % Identification of each agent type
  
  % Artist Manager -> 1 ArtistManagerAgent
  % Curator -> + CuratorAgent
  % Profiler -> 1 ProfilerAgent
  
  Agent model is describe below in Figure~\ref{figure:agent_model}.
  
\tikzstyle{block} = [rectangle, draw, %fill=blue!20, 
    text width=8em, text centered, rounded corners, minimum height=2em]
\tikzstyle{line} = [draw, -latex']
    
\begin{figure}[ht!]
\begin{tikzpicture}[node distance = 4cm, auto]
    % Place nodes
    \node [block] (ama) {ArtistManager Agent};
    \node [block, above of=ama, node distance=2cm] (am) {ArtistManager};
    
    \node [block, right of=ama] (cua) {Curator Agent};
    \node [block, above of=cua, node distance=2cm]  (cu) {Curator};

    \node [block, right of=cua] (proa) {Profiler Agent};
    \node [block, above of=proa, node distance=2cm] (pro) {Profiler};
    
    \path [line, ] (am) -- node {1}(ama);
    \path [line] (cu) -- node {+}(cua);
    \path [line] (pro) -- node {1}(proa);
\end{tikzpicture}
\caption{Agent model}
\label{figure:agent_model}
\end{figure}

  
  \subsection{Service Model}
  % TODO for each protocol
  % Service | Input | Output | Preconditions | Postconditions
  
  \newcommand*{\thead}[1]{\multicolumn{1}{c}{\bfseries #1}}
  
  \begin{table}
  \begin{tabular}{|l|l|l|l|l|}
  \thead{Service} & \thead{Input} & \thead{Output} & \thead{Preconditions} & \thead{Postconditions} \\\hline
  - & - & - & - & - \\ \hline
  - & - & - & - & - \\ \hline
  \end{tabular}
  
  \caption{Service model}
  \label{table:service_model}
  \end{table}

  
  % DutchAuction (ArtistManagerAgent <-> CuratorAgent(s))
  % ExploreProducts (ProfilerAgent <-> CuratorAgent(s))
  % BuyProduct (ProfilerAgent <-> CuratorAgent)
  
  \subsection{Acquaintance Model}
  % TODO Show the communication links between the agents
  
  
  
  % ArtistManagerAgent <-> CuratorAgent(s) <-> Profiler
  
  \subsection{Mobility Model}
  % TODO
  % a. Identification of place types (example: Museo Galileo
  % Museum).
  % b. Identification of relationships between agent types and
  % place types
  % c. Definition of the cardinality between agent types and
  % place types
  % d. Identification of the travel path of each mobile agent
  
  % CuratorAgent(s) move to ArtistManagerAgent place!

  \subsubsection{Place types}
  % TODO
  
  \subsubsection{Relationships}
  % TODO
  
  \subsubsection{Cardinality}
  % TODO  
  
  \subsubsection{Travel paths}
  % TODO
  
  
  %%%%%%
  
  
  \section{Agent Interaction Protocols} % task 2
  % Model interactions among agents in AgentUML
  
  
  \subsection{Overall protocol} % level 1
  % TODO Output: Detailed package and template diagrams 
  %              + brief description
  
  
  \subsection{Interactions among agents}
  % TODO Output: Sequence diagrams + brief description
  
  
  \subsection{Internal agent processing}
  % TODO Output: State chart diagrams + brief description


  %%%%%%
  
  
  \section{Behaviors design} % task 3
  % UML diagrams to design behavior of your agents
  
  % TODO UML class diagrams according to the descriptions 
  %      found in the reference

  
  %%%%%%
  
  \clearpage % TODO to remove
  
  \section{Role-based modeling} % task 4
  % Model your system using "Role based modeling" approach
  
  \subsection{Modeling}
  % TODO Task 4.1 Perform role-based modeling using
  %               RoMAS for the initial task.	
  
  \subsection{Comparison with GAIA}
  % TODO Task 4.2 Comment on differences in resulting
  %               designs of 4.1 and GAIA (from Task 1).
  % (i.e. Analysis phase of GAIA against performing 
  %  role-based modeling as first step to GAIA analysis)

  
  %%%%%%
  
  
  \section{JADE and other agent platforms} % task 5
  % There are number of implementations of agent platforms which conform to 
  % FIPA specifications. Perform a comparison of at least 2 other agent 
  % platforms with JADE
  
  % Some other platforms: Agent Development Kit, FIPA-OS, 
  %                       Jack Intelligent Agents, ZEUS, 
  %                       SAGE, ...
  % TODO choose 2 of them
  
  \subsection{platform 1} % TODO to choose
  
  \subsubsection{Architecture}
  % TODO Architecture of platform
  
  \subsubsection{Services provided}
  % TODO Services provided by platform
  
  \subsubsection{Simple scenario}
  % TODO Comparison of implementation of a simple scenario same as Question 2 (??)
  % (i.e. service implementation, service registration and service discovery)
  
  \subsubsection{Notable projects}
  % TODO List some notable projects which use that platform
  
  \subsubsection{Personnel opinion}
  % TODO Your personnel opinion/judgment about the platform as compared to JADE
  % Yo can take part 3 (simple scenario) as your starting point, and explain the 
  % architecture and services the platform provides from a practical point of view
  
  \subsection{platform 2} % TODO to choose 
  
  \subsubsection{Architecture}
  % TODO Architecture of platform
  
  \subsubsection{Services provided}
  % TODO Services provided by platform
  
  \subsubsection{Simple scenario}
  % TODO Comparison of implementation of a simple scenario same as Question 2 (??)
  % (i.e. service implementation, service registration and service discovery)
  
  \subsubsection{Notable projects}
  % TODO List some notable projects which use that platform
  
  \subsubsection{Personnel opinion}
  % TODO Your personnel opinion/judgment about the platform as compared to JADE
  % Yo can take part 3 (simple scenario) as your starting point, and explain the 
  % architecture and services the platform provides from a practical point of view
  
  
  %%%%%%
  
  
  \section{Conclusion}
  % TODO what this project gave us ?

  
% bibliography
%\bibliographystyle{plainurl}
%\bibliography{references}

\end{document}