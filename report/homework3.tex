\documentclass[a4paper,11pt]{article}
\usepackage[utf8]{inputenc}
\usepackage[english]{babel}
\usepackage[font=small,labelfont=bf, textfont=it]{caption}
\usepackage[bottom=2cm, left=3.7cm, right=3.7cm]{geometry} % to change padding
\usepackage{verbatim}
\usepackage{subcaption} % for multi figure
\usepackage{enumitem} % -- label item
\usepackage{tabularx}
\usepackage{color}
\usepackage[usenames, dvipsnames]{xcolor} % color
\usepackage[framemethod=TikZ]{mdframed} % box
\usepackage{listings} % code
\usepackage{amsmath} % align* maths
\usepackage{wrapfig}
\usepackage[bookmarks, hidelinks]{hyperref} % clickable links
\usepackage{graphicx} % includegraphics
\usepackage[section]{placeins} % float inside section
\usepackage{indentfirst}

\setlength\parindent{0pt}
\setlength{\parskip}{1em}



% Titre
\title{Distributed Artificial Intelligence \& Intelligent Agents \\ DAIIA - Homework 3}
\author{Laurentiu CAPATINA \& Quentin LEMAIRE}

\begin{document}

  \maketitle % build title

  \section{Introduction}
  
  The aim of this homework is to implement practical scenarios using deep 
mechanisms of the JADE\footnote{Java Agent DEvelopment: \href{http://jade.tilab.com/}{http://jade.tilab.com/}} 
  platform. Such mechanisms involve knowledge and understanding of containers and self 
  generation of agent's concepts.

  % TODO to complete

  % aliases
  \newcommand{\pa}{\textit{Profiler Agent}}
  \newcommand{\cu}{\textit{Curator Agent}}
  \renewcommand{\to}{\textit{Tour Guide Agent}}
  \newcommand{\am}{\textit{Artist Manager Agent}}
  
  \section{Implementation}
  
  \subsection{N queens}

  \subsubsection{Description}
  
  % TODO what's the problem ?
  \textit{N queens problem} is a well-known problem in Artificial Intelligence because it is often 
  used to demonstrate that problems that, at first glance, seem difficult to solve are not that 
  hard to solve if we concentrate and study the possible issues of the problem.
  
  \subsubsection{Protocol}
  
  \paragraph{Model}
  Before building any communication protocol betweens the Queens, it was necessary to implement a 
  small model in order to create the rules of the game and the chest where to Queens stand. That's 
  why a class called \textit{Chest} appeared in the model. The aim of this class is to provide 
  methods to help Queens in their decision making. First, the chest was virtually implemented as a 
  single table (list) of integers from 0 to $n^2-1$. Of course, such a table isn't useful at all and 
  we only created a table representing the position of the $n$ queens. This table contains $n$ entries 
  and each $i$ of them contains the value (position) of the queen number $i$.
  \begin{verbatim}
	n = 4
      ===========
      00 01 02 03
      04 05 06 07
      08 09 10 11
      12 13 14 15
      ===========
queens = [04, 13, 02, 11]
  \end{verbatim}

  Methods provided are the following:
  \begin{itemize}
   \item serialize: in order to send chest through content ACLMessage
   \item fromSerial: build a chest from a serialize content
   \item setQueen: set queen $i$ to a position $p$
   \item findPlace: find place within a column
   \item findPlaceFromPrevious: find place without taking previous positions of the queen into account
   \item possible: is it possible to place a queen here ?
   \item checkPosition \& check are private methods which allow checking position parameter and position 
   compared to other queens (same row ? same column ? same diagonal ?)
  \end{itemize}

  \paragraph{Communication}  
  Each queen represents a column in the chest and could only communicate with the previous or the following queen (column before and column after). 
  For the first queen, she can only talk to the next one (second column), and for the last queen, she can only speak with the previous queen. Once every 
  queens are launched (created dynamically with \textit{ContainerController} class in \textit{QueenFactoryAgent} class), they look for their previous 
  and/or next queen(s) using registrations and subscriptions to and from \textit{Directory Facilitator} (DF). Once this is done, a backtracking process 
  can start with the following protocol:
  \begin{enumerate}
   \item First queen (column number 0) starts with positioning in the first column (often first line)
   \item First queen sends a message to next queen asking her to position herself if she can
   \item Second queen receives the message with associated model and try to place herself on the chest
   \begin{itemize}
    \item If she succeeds, she sends a message to the following queen and then step 2 is repeated.
    \item If she fails, she sends a message back to the previous queen asking her to change the position of the queen.
   \end{itemize}
   \item If a queen receives a message from the next queen, it means ``change your position'' i.e. no solution was found.
   Queen receiving the message tries to place her queen in another position (not forgetting the previous position in order 
   to avoid infinite loops). Then, conditions and actions from step 3 are repeated.
   \item If last queen succeeds in placing herself, problem is solved
   \item If first queen doesn't succeed in placing all of her position, problem is impossible to solve or every solution has 
   been found.
  \end{enumerate}
  
  \paragraph{Some results}
\begin{verbatim}
n = 04 | solution = [04, 13, 02, 11]
n = 08 | solution = [0, 33, 58, 43, 20, 53, 14, 31]
n = 16 | solution = [0, 33, 66, 19, 196, 133, 214,
                     183, 232, 89, 250, 107, 60, 
                     173, 126, 159]
n = 20 (take a long time to compute)
XXX 001 002 003 004 005 006 007 008 009 010 011 012 013 014 015 016 017 018 019
020 021 022 XXX 024 025 026 027 028 029 030 031 032 033 034 035 036 037 038 039
040 XXX 042 043 044 045 046 047 048 049 050 051 052 053 054 055 056 057 058 059
060 061 062 063 XXX 065 066 067 068 069 070 071 072 073 074 075 076 077 078 079
080 081 XXX 083 084 085 086 087 088 089 090 091 092 093 094 095 096 097 098 099
100 101 102 103 104 105 106 107 108 109 110 111 112 113 114 115 116 117 XXX 119
120 121 122 123 124 125 126 127 128 129 130 131 132 133 134 135 XXX 137 138 139
140 141 142 143 144 145 146 147 148 149 150 151 152 153 XXX 155 156 157 158 159
160 161 162 163 164 165 166 167 168 169 170 XXX 172 173 174 175 176 177 178 179
180 181 182 183 184 185 186 187 188 189 190 191 192 193 194 XXX 196 197 198 199
200 201 202 203 204 205 206 207 208 209 210 211 212 213 214 215 216 217 218 XXX
220 221 222 223 224 225 226 XXX 228 229 230 231 232 233 234 235 236 237 238 239
240 241 242 243 244 XXX 246 247 248 249 250 251 252 253 254 255 256 257 258 259
260 261 262 263 264 265 266 267 268 269 270 271 272 273 274 275 276 XXX 278 279
280 281 282 283 284 285 XXX 287 288 289 290 291 292 293 294 295 296 297 298 299
300 301 302 303 304 305 306 307 308 309 310 311 XXX 313 314 315 316 317 318 319
320 321 322 323 324 325 326 327 328 329 XXX 331 332 333 334 335 336 337 338 339
340 341 342 343 344 345 346 347 XXX 349 350 351 352 353 354 355 356 357 358 359
360 361 362 363 364 365 366 367 368 369 370 371 372 XXX 374 375 376 377 378 379
380 381 382 383 384 385 386 387 388 XXX 390 391 392 393 394 395 396 397 398 399
\end{verbatim}

For each problem, we could have searched all the possible solutions backtracking even 
though we had found a solution in order to find it all (stacking them).
  
Concerning time, this algorithm is really huge resources consuming. There are lots of 
communication and backtracking before finding a solution with $n = 16$. This is way much 
easier with $n = 4$ but this problem doesn't seem to be solvable in that way for big $n$ 
values. Even $n = 100$ seems very very hard nay impossible to compute in an affordable 
time for a computer machine.
  
  % TODO Lancement  -gui main:Agents.QueenFactoryAgent

  \subsection{Multiple containers}

  \subsubsection{Description}

  The idea of this task is to create a distributed Dutch auction between the Artist Manager
  and the Curators in 2 different containers.
  We have chosen the following scenario:
  \begin{itemize}
  \item The Artist Manager is \textit{cloned} in two different containers
  \item Four Curators are created with different strategies associated
  \item Each Curators is \textit{moved} in one of the two containers 
  \item An auction takes place in each container
  \item The main Artist Manager receives the two auction results and keeps the best one
  \end{itemize}

  \subsubsection{Implementation}

  \paragraph{Controller Agent}

  An introduction of a new Agent was necessary in order to handle the creation of containers
  as well as the identification of all possible locations on the platform. Moreover, using the
  subscription in the DF Agent, it manages to get all Artist Manager Agents in order to
  launch their cloning. It will also create the four curators and will take care of their
  moving process.

  \paragraph{Cloning}

  This process gives the oportunity to have an identical agent in another container with a different
  name. Consequently, it is based on the \textit{CloneAction} and \textit{MobileDescriptionClass} classes.
  The main parameters are the new destination and the new name of the agent.
  In order to execute the cloning, an \textit{ACLMessage} is sent to the corresponding instance of the agent
  with the performative \textit{REQUEST}. This message has a specific language (\textit{SLCodec}) and a specific
  ontology (\textit{MobilityOntology}).

  \paragraph{Moving}

  Moving is a mobility very similar to cloning. The difference is that the agent won't be present in
  two different containers, it will simple move from one to a new destination. All other aspects mentioned
  in the case of cloning are valid for moving as well.

  \paragraph{Distributed Auction}

  The main idea is to apply the auction but only to a subset of all Curators. Consequently, this implies
  applying a filter on the research of all Curators before launching the auction. The \textit{destination}
  of the moved Curators may serve as filter while searching the services of Curators in the DF Agent. It is
  mandatory that the mobility precedes the DF research so that the two behaviours would be sequential.

  \paragraph{Main Artist Manager}
  This specific Artist Manager which is in the main container shouldn't launch an auction. It should listen
  to the other two Artist Managers for the winner of their actions in their containers and he will eventually
  choose the best offer. The communication will be done using \textit{ACLMessages} with a \textit{block()}
  function at the level of the main Artist Manager in order to capture the two results. This part
  of the implementation still needs to be tested as it is not fully functional.

  \section{Conclusion}
  
  This homework was an opportunity to implement the N Queen problems using agents based on the JADE framework
  but also to implement the mobility of Curators and Artist Manager Agents in order to create a distributed
  Dutch auction.

\end{document}
