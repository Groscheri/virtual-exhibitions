\documentclass[a4paper,11pt]{article}
\usepackage[utf8]{inputenc}
\usepackage[english]{babel}
\usepackage[font=small,labelfont=bf, textfont=it]{caption}
\usepackage[bottom=2cm, left=3.7cm, right=3.7cm]{geometry} % to change padding
\usepackage{verbatim}
\usepackage{subcaption} % for multi figure
\usepackage{enumitem} % -- label item
\usepackage{tabularx}
\usepackage{color}
\usepackage[usenames, dvipsnames]{xcolor} % color
\usepackage[framemethod=TikZ]{mdframed} % box
\usepackage{listings} % code
\usepackage{minted} % code
\usepackage{amsmath} % align* maths
\usepackage{wrapfig}
\usepackage[bookmarks, hidelinks]{hyperref} % clickable links
\usepackage{graphicx} % includegraphics
\usepackage[section]{placeins} % float inside section
\usepackage{indentfirst}

\setlength\parindent{0pt}
\setlength{\parskip}{1em}



% Titre
\title{Distributed Artificial Intelligence \& Intelligent Agents \\ DAIIA - Homework 1}
\author{Laurentiu CAPATINA \& Quentin LEMAIRE}

\begin{document}

  \maketitle % build title

  \section{Introduction}
  
  The aim of this homework is to discover the JADE\footnote{Java Agent DEvelopment: \href{http://jade.tilab.com/}{http://jade.tilab.com/}} platform. 
  Through a virtual exhibitions scenario, we will implement agents and behaviors which will let us understand how the framework is working 
  and how we can create a multi-agent system.
  
  
  \subsection{Agents}
  
  In this homework, 3 different agents were implemented according to the scenario of the museum:
  \begin{itemize}[label=--]
   \item \textbf{Profiler Agent}: this is the user agent, he can access every information of the user (age, occupation, interest, ...)
   \item \textbf{Tour Guide Agent}: this is the smart museum agent which build virtual tour for the profiler agent according to interest of the user
   \item \textbf{Curator Agent}: this is the museum manager, he can access every information about the gallery's artifacts
  \end{itemize}

  
  \subsection{Behaviors}
  
  Behaviors corresponds to actions an agent can achieve. An agent can belong several behaviors at the same time. In order to improve our knowledge of 
  the framework, two kinds of behaviors were implemented:
  \begin{itemize}[label=--]
   \item \textbf{5 Simple Behaviors}: OneShotBehavior, SimpleAchieveREInitiator, SimpleAchieveREResponder, WakerBehavior, TickerBehavior
   \item \textbf{2 Composite Behaviors}: ParallelBehavior, SequentialBehavior
  \end{itemize}
  
  % aliases
  \newcommand{\pa}{\textit{Profiler Agent}}
  \newcommand{\cu}{\textit{Curator Agent}}
  \renewcommand{\to}{\textit{Tour Guide Agent}}
  
  \section{Implementation}
  
  % TODO small introduction
  
  \subsection{Profiler Agent}
  
  Some model has been implemented in order to add concrete properties to the \pa{}. A \textit{User} class has been implemented containing information 
  about the user such as name, age, gender, interest, occupation, \dots{}
  
  During the initialization of the agent, several things are happening. First, all parameters from the command line are retrieved and analyzed in order 
  to get information about the \cu{}'s name and the interests of the user. Then a self implemented behavior -- extending JADE \textit{SimpleAchieveREInitiator} 
  class -- called \textit{RequestCurator} is created in order to request the \cu{}. The communication is made through \textit{ACLMessage} with the \textit{REQUEST} 
  FIPA Performative.
  
  In order to remain as close as possible to the scenario, a \textit{WakerBehavior} has been implemented at the creation of the agent. After 5 seconds, 
  all setup actions are launched. This latency corresponds to the loading time of the mobile application.
  
  The mobile application represented by this agent has to be displayed. That's why a \textit{SequentialBehavior} has been implemented with the following 
  sub-behaviors:
  \vspace{-15pt}
  \begin{enumerate}
   \item A behavior to display the list of interests of the user (\textit{OneShotBehavior})
   \item A behavior to display the button to ask for a virtual tour (\textit{OneShotBehavior})
   \item A behavior to display the button to get contact information (\textit{OneShotBehavior})
  \end{enumerate}
  
  \subsection{Tour Guide Agent}
  
  % Propose un service de création de virtualTour : Building-virtual-tour
  
  \to{} also has a \textit{RequestCurator} behavior in order to request the \cu{}. This agent is responsible to build virtual tour 
  according to a list of interests (sent by the \pa{}). That's why it contains a \textit{BuildVirtualTour} behavior (extending \textit{OneShotBehavior}) 
  has been implemented. This behavior takes the list of interests of the user and requests the \cu{} to compute 
  the virtual tour. It is also possible to give a specific name to the created virtual tour.
  
  \subsection{Curator Agent}
  
  The \cu{} corresponds to the ``database'' of the museum. It can access all the artifacts of the museum which are represented as a list 
  of \textit{Item}s which contain different properties: id, name, creator, creation, place, genre.
  
  A \textit{TickerBehavior} called \textit{UpdateGallery} has been implemented in order to update the gallery items each 5 seconds. This 
  behavior corresponds to the update (buy or sell) of products within the gallery.
  
  Meanwhile, a \textit{ParallelBehavior} has been implemented for the purpose of this agent. Two sub-behaviors were created in order to 
  listen to requests of \to{} and \pa{}. Those behaviors were implemented extending \textit{SimpleAchieveREResponder} behavior.

%   \vspace{-15pt}
  % [linenos, frame=lines]
%   \begin{minted}{java}
%     protected long id;
%     protected String name;
%     protected String creator;
%     protected Date creation;
%     protected String place;
%     protected String genre;
%   \end{minted}


  
  \subsection{Directory Facilitator (DF)}

  % TODO small introduction to what is DF ?
  
  \subsubsection{Register} % Obj-complementary-info (Curator) & Building-virtual-tour (TourGuide)
  
  % TODO to complete
  
  The \cu{} and the \to{} both provide one service to DF. The \cu{} registers a service as \textbf{Obj-complementary-info}.
  The aim of this service is to provide information about an artifact belonging to the museum gallery.
  % TODO add more information about the service ? (parameters ?)
  The \to{} registers a service as \textbf{Building-virtual-tour}. This service retrieves interests of an user and requests 
  the \cu{} in order to build a virtual tour corresponding to the given interests.
  % TODO add more information about the service ? (name of the virtual tour ?)
  
  \subsubsection{Discover} % search (behavior searchAllServices)
  The \pa{} wants to use a service but this agent doesn't know where it can find it and how to use it. That's why the agent needs 
  to request the DF to discover all available (registered) services. Then, this agent can choose which service to use and DF will 
  answer it with parameters needed to use the service.
  
  A behavior called \textit{searchAllServices} has been implemented in order to let the \pa{} look for services information.
  
  % TODO to complete
  
  \subsubsection{Subscribe} % TODO what is this ???
  
  \section{Conclusion}
  
  This homework helped us to understand how JADE works and how to implement a multi-agent system using this framework. The given 
  scenario let us create a coherent environment.
  
\end{document}
