\documentclass[a4paper,11pt]{article}
\usepackage[utf8]{inputenc}
\usepackage[english]{babel}
\usepackage[font=small,labelfont=bf, textfont=it]{caption}
\usepackage[bottom=2cm, left=3.7cm, right=3.7cm]{geometry} % to change padding
\usepackage{verbatim}
\usepackage{subcaption} % for multi figure
\usepackage{enumitem} % -- label item
\usepackage{tabularx}
\usepackage{color}
\usepackage[usenames, dvipsnames]{xcolor} % color
\usepackage[framemethod=TikZ]{mdframed} % box
\usepackage{listings} % code
\usepackage{amsmath} % align* maths
\usepackage{wrapfig}
\usepackage[bookmarks, hidelinks]{hyperref} % clickable links
\usepackage{graphicx} % includegraphics
\usepackage[section]{placeins} % float inside section
\usepackage{indentfirst}

\setlength\parindent{0pt}
\setlength{\parskip}{1em}



% Titre
\title{Distributed Artificial Intelligence \& Intelligent Agents \\ DAIIA - Homework 1}
\author{Laurentiu CAPATINA \& Quentin LEMAIRE}

\begin{document}

  \maketitle % build title

  \section{Introduction}
  
  The aim of this homework is to discover the JADE\footnote{Java Agent DEvelopment: \href{http://jade.tilab.com/}{http://jade.tilab.com/}} platform. 
  Based on a virtual exhibition scenario, we will implement agents and behaviors which will give us a deep understanding of the JADE framework 
  and of the creation of a multi-agent system.
  
  
  \subsection{Agents}
  
  In this homework, 3 different agents were implemented according to the scenario of the museum:
  \begin{itemize}[label=--]
   \item \textbf{Profiler Agent}: this is the user agent, he can access information of the user (age, occupation, interest, ...)
   \item \textbf{Tour Guide Agent}: this is the smart museum agent which builds virtual tours for profiler agents according to their interest
   \item \textbf{Curator Agent}: this is the museum manager, he can access information about the gallery's artifacts
  \end{itemize}

  
  \subsection{Behaviors}
  
  Behaviors correspond to actions an agent can achieve. An agent has several behaviors. Two types of behaviors were implemented:
  \begin{itemize}[label=--]
   \item \textbf{5 Simple Behaviors}: OneShotBehavior, SimpleAchieveREInitiator, SimpleAchieveREResponder, WakerBehavior, TickerBehavior
   \item \textbf{3 Composite Behaviors}: ParallelBehavior, SequentialBehavior, SubscriptionInitiator
  \end{itemize}
  
  % aliases
  \newcommand{\pa}{\textit{Profiler Agent}}
  \newcommand{\cu}{\textit{Curator Agent}}
  \renewcommand{\to}{\textit{Tour Guide Agent}}
  
  \section{Implementation}
  
  The virtual exhibition is defined by the 3 types of agents previously mentioned and who communicate between them as following: the \pa{} sends its
  interests which will be used by the \to{} for building a personalised tour. After receiving a virtual tour, the \pa{} asks 
  the \cu{} complementary information about certain objects. Based on this simple scenario, several behaviors were implemented. The aim was not to create an
  exhaustive implementation but to have certain parts of the scenario which outline the key behaviors.
  
  \subsection{Profiler Agent}
  
  A model has been implemented in order to add specific properties to the \pa{}. A \textit{User} class has been created containing information 
  such as name, age, gender, interest, occupation, \dots{}
  
  During initialization, several steps were implemented. Firstly, parameters from the command line are retrieved in order 
  to get information about the \cu{}'s interests. Then a self implemented behavior called \textit{RequestCurator} extending JADE \textit{SimpleAchieveREInitiator} 
  class was created in order to request the \cu{}. The communication is made through \textit{ACLMessage} with the \textit{REQUEST} 
  FIPA Performative.
  
  In order to remain as close as possible to the scenario, a \textit{WakerBehavior} has been implemented for the profile agent. After 5 seconds, 
  all setup actions are launched. This latency corresponds to the loading time of the mobile application.
  
  The mobile application represented by this agent has to be displayed. That's why a \textit{SequentialBehavior} has been implemented with the following 
  sub-behaviors:
  \vspace{-15pt}
  \begin{enumerate}
   \item A behavior to display the list of interests of the user (\textit{OneShotBehavior})
   \item A behavior to display the button to ask for a virtual tour (\textit{OneShotBehavior})
   \item A behavior to display the button to get contact information (\textit{OneShotBehavior})
  \end{enumerate}
  
  \subsection{Tour Guide Agent}
  
  % Propose un service de création de virtualTour : Building-virtual-tour
  
  Like the previous agent, \to{} also has a \textit{RequestCurator} behavior in order to request the \cu{}. This agent is responsible for building virtual tours 
  according to a list of interests (sent by the \pa{}). That's why it contains a \textit{BuildVirtualTour} behavior (extending \textit{OneShotBehavior}). 
  This behavior takes the list of interests of the user and requests the \cu{} to compute the virtual tour. It is also possible to give a 
  specific name to the created virtual tour.
  
  \subsection{Curator Agent}
  
  The \cu{} could be seen as the ``database'' of the museum. It can access all the artifacts of the museum which are represented as a list 
  of \textit{Item}s which contain different properties: id, name, creator, creation, place, genre.
  
  A \textit{TickerBehavior} called \textit{UpdateGallery} has been implemented in order to update the gallery items each 5 seconds. This 
  behavior corresponds to the update (buy or sell) of products within the gallery.
  
  Moreover, a \textit{ParallelBehavior} has been implemented for this agent. Two sub-behaviors were created in order to 
  listen to requests of \to{} and \pa{}. Those behaviors were implemented extending \textit{SimpleAchieveREResponder} behavior.

%   \vspace{-15pt}
  % [linenos, frame=lines]
%   \begin{minted}{java}
%     protected long id;
%     protected String name;
%     protected String creator;
%     protected Date creation;
%     protected String place;
%     protected String genre;
%   \end{minted}


  
  \subsection{Directory Facilitator (DF)}

  The DF can be seen as a sort of ``Yellow pages''. Each FIPA compliant platform already hosts a default DF agent. In other words, the DF agent
  contains a centralised list of services offered by other agents on the platform. In order for a service to be on this specific list, 
  the agent has to register it so that other agents could later identify it.
  
  \subsubsection{Register} % Obj-complementary-info (Curator) & Building-virtual-tour (TourGuide)
  The \cu{} and the \to{} both provide one service to DF. The \cu{} registers a service as \textbf{Obj-complementary-info}.
  The aim of this service is to provide complementary information about an artifact belonging to the museum gallery. Moreover, it has a type
  of \textbf{object-info} which will enable to be easier found in the DF agent. According to the scenario, it should be called 
  by the \pa{} after receiving a guided tour from the \to{}. The later registers a service as well called \textbf{Building-virtual-tour}. 
  This service retrieves interests of an user and requests the \cu{} in order to build a virtual tour corresponding to the given interests.
  A type was also associated to this service in order to facilitate the research of it: \textbf{build-tour}.
  
  On the other hand, the registered services are deleted from the DF agent at the moment when the agent which offers the specific service is deleted.
  This is done in the \textit{takeDown} method of an agent and the process is called deregistration.
  
  \subsubsection{Discover} % search (behavior searchAllServices)
  The \pa{} for example wants to use a service but the agent doesn't know where it can find it and how to use it. That's why the agent needs 
  to request the DF to discover all available (registered) services. Then, this agent can choose which service to use and DF will 
  return the parameters needed to call the service. The research in the DF can be done by using templates or constraints. Moreover,
  there is also the possibility to use a blocking research called \textit{searchUntilFound} which waits for a returned service in order to continue execution.
  
  A behavior called \textit{searchAllServices} has been implemented in order to give the \pa{} the opportunity to search for services. This behavior
  enables the possibility to choose a service among those displayed in the console and the parameters needed will be displayed in the standard output as well.
  In this case, as the \pa{} wants all services, we use neither a template nor constraints.
  
  \subsubsection{Subscribe}
  The subscription process works as following: an agent sends a create subscription message to the DF agent and it will be notified once a
  service corresponding to its demand will pe present in the DF. We used this process for the \pa{}. The main goal for this agent is to be
  able to ask the \to{} to build him a virtual tour. Consequently, the \pa{} uses subscription to be informed when a DF service of type
  \textbf{build-tour} exists in the DF. For the purpose of initializing this communication via subscription, a composite behavior called
  \textit{SubscriptionInitiator} was used which sends a \textit{createSubscriptionMessage} to the DF agent. When a corresponding service is
  registered, the \pa{} is notified.
  
  \section{Conclusion}
  
  This homework helped us understand how JADE works and how to implement a multi-agent system using this framework. The given 
  scenario gave us the possibility to create a simple, multi-agent and multi-behavior environment.
  
\end{document}
