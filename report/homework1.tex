\documentclass[a4paper,11pt]{article}
\usepackage[utf8]{inputenc}
\usepackage[english]{babel}
\usepackage[font=small,labelfont=bf, textfont=it]{caption}
\usepackage[bottom=2cm, left=3.7cm, right=3.7cm]{geometry} % to change padding
\usepackage{verbatim}
\usepackage{subcaption} % for multi figure
\usepackage{enumitem} % -- label item
\usepackage{tabularx}
\usepackage{color}
\usepackage[usenames, dvipsnames]{xcolor} % color
\usepackage[framemethod=TikZ]{mdframed} % box
\usepackage{listings} % code
\usepackage{minted} % code
\usepackage{amsmath} % align* maths
\usepackage{wrapfig}
\usepackage[bookmarks, hidelinks]{hyperref} % clickable links
\usepackage{graphicx} % includegraphics
\usepackage[section]{placeins} % float inside section
\usepackage{indentfirst}

\setlength\parindent{0pt}
\setlength{\parskip}{1em}



% Titre
\title{Distributed Artificial Intelligence \& Intelligent Agents \\ DAIIA - Homework 1}
\author{Laurentiu CAPATINA \& Quentin LEMAIRE}

\begin{document}

  \maketitle % build title

  \section{Introduction}
  
  The aim of this homework is to discover the JADE\footnote{Java Agent DEvelopment: \href{http://jade.tilab.com/}{http://jade.tilab.com/}} platform. 
  Through a virtual exhibitions scenario, we will implement agents and behaviors which will let us understand how the framework is working 
  and how we can create a multi-agent system.
  
  
  \subsection{Agents}
  
  In this homework, 3 different agents were implemented according to the scenario of the museum:
  \begin{itemize}[label=--]
   \item \textbf{Profiler Agent}: this is the user agent, he can access every information of the user (age, occupation, interest, ...)
   \item \textbf{Tour Guide Agent}: this is the smart museum agent which build virtual tour for the profiler agent according to interest of the user
   \item \textbf{Curator Agent}: this is the museum manager, he can access every information about the gallery's artifacts
  \end{itemize}

  
  \subsection{Behaviors}
  
  Behaviors corresponds to actions an agent can achieve. An agent can belong several behaviors at the same time. In order to improve our knowledge of 
  the framework, two kinds of behaviors were implemented:
  \begin{itemize}[label=--]
   \item \textbf{5 Simple Behaviors}: OneShotBehavior, SimpleAchieveREInitiator, SimpleAchieveREResponder, WakerBehavior, TickerBehavior
   \item \textbf{2 Composite Behaviors}: ParallelBehavior % TODO add another composite behavior
  \end{itemize}
  
  % aliases
  \newcommand{\pa}{\textit{Profiler Agent}}
  \newcommand{\cu}{\textit{Curator Agent}}
  \renewcommand{\to}{\textit{Tour Guide Agent}}
  
  \section{Implementation}
  
  % TODO small introduction
  
  \subsection{Profiler Agent}
  
  Some model has been implemented in order to add concrete properties to the \pa{}. A \textit{User} class has been implemented containing information 
  about the user such as name, age, gender, interest, occupation, \dots{}
  
  During the initialization of the agent, several things are happening. First, all parameters from the command line are retrieved and analyzed in order 
  to get information about the \cu{}'s name. Then a self implemented behavior -- extending JADE \textit{SimpleAchieveREInitiator} class -- called 
  \textit{RequestCurator} is created in order to request the \cu{}. The communication is made through \textit{ACLMessage} with the \textit{REQUEST} FIPA 
  Performative.
  
  % TODO ProfilerAgent : WakerBehavior wakes up after 5 seconds => correspond to time of loading the app
  
  \subsection{Tour Guide Agent}
  
  % TODO
  
  \to{} also has a \textit{RequestCurator} behavior in order to request the \cu{}.
  
  \subsection{Curator Agent}
  
  % TODO
  % TODO Item / ListItem Model
  % TODO Update Gallery (TickerBehavior)
  
  \subsection{Directory Facilitator (DF)}

  % TODO small introduction to what is DF ?
  
  \subsubsection{Register}
  
  % TODO to complete
  The \cu{} and the \to{} both provide one service to DF. The \cu{} registers a service as \textbf{XXXX}. % TODO ADD NAME OF THE SERVICE
  % TODO ADD INFORMATION ABOUT THE SERVICE HERE
  The \to{} registers a service as \textbf{XXXXX}. % TODO ADD NAME OF THE SERVICE
  % TODO ADD INFORMATION ABOUT THE SERVICE HERE
  
  \subsubsection{Discover}
  % TODO to complete
  The \pa{} wants to use a service but this agent doesn't know where it can find it and how to use it. That's why the agent needs 
  to request the DF to discover all available (registered) services. Then, this agent can choose which service to use and DF will 
  answer it with parameters needed to use the service.
  
  \subsubsection{Subscribe} % TODO what is this ???
  
\end{document}
